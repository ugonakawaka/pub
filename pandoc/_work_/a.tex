\documentclass[11pt,a4paper]{jsarticle}

\usepackage{booktabs}
\usepackage{multirow}
\usepackage[dvipdfmx]{graphicx}
\usepackage{float}
\usepackage{url}
\usepackage{amsmath}
\usepackage{array}
\usepackage{plistings}
\usepackage[top=35truemm,bottom=30truemm,left=30truemm,right=30truemm]{geometry}

 

\begin{document}
\section{目的}
ここに目的を書くよ

\section{原理}
原理はこんな感じです。
改行しても改行されませんね。こんな風にたくさん書いたり、変なところで改行
しても出力は問題
なく行われているのがわかると思います。

こんな風に1行スペースを開けると改行されて先頭が1文字分下がります。
\section{実験器具}
\begin{itemize}
\item 実験器具1
\item はんだごて
\item こんな感じです
\end{itemize}

\section{方法}
\subsection{画像の貼り方}
ここでは画像の貼り方の説明します。
画像ファイル(ここではtest.JPG)は、今書いてるTeXファイルと同じフォルダ内に置かないとエラーになるので気を付けてください。
\begin{figure}[htbp]
\centering
\includegraphics[width=13cm]{test.JPG}
\caption{Texstudio}
\end{figure}

\begin{figure}[htbp]
\centering
\includegraphics[width=5cm]{test.JPG}
\caption{名前を書くよ}
\end{figure}
画僧の貼り方はいろいろなやり方があるみたいですが自分はこれしか知りません。
captionのところに名前を付けてあげれば図1、とかは勝手に番号振ってくれます。
widthの値を変えると大きさが変わります。
[htpb]はおまじないでいいと思います。気になる方は調べてみてください。

\subsection{数式の書き方}
書くの遅くね?って気がするけど数式系のお話です。
文章のなかで数式等を書く場合にはドルマークでサンドウィッチします。
例えば$y=ax+b$みたいな感じですね。何も書かないとy=ax+bみないな感じ。
文字のフォントが違うのがわかるでしょう。
そんな感じで行を変えて真ん中に書くのはこれ。TeXstudioだとcnrl+shift+Nで出てきます。
\begin{equation}\label{key}
y=ax+b
\end{equation}
番号をつけない数式はこれ
\[
y=ax+b
\]
その他$\sum$とか$\lim$とか$\int_{t=0}^{t=\infty}$とか$\pi$とかは各々必要な時に調べてみてください。
\section{結果}
次に表の書き方についてまとめます。
以下のような書き方で表が作成できます。
% Table generated by Excel2LaTeX from sheet 'Sheet1'
\begin{table}[htbp]
\centering
\caption{ここに名前を書くよ}
\begin{tabular}{ccc}
\toprule
\multicolumn{1}{l}{時間} & \multicolumn{1}{l}{温度} & \multicolumn{1}{l}{制御量} \\
\midrule
0 & 23 & \\
0.5 & 23 & 3.2 \\
1 & 23 & 12.4 \\
1.5 & 23 & 21.8 \\
2 & 23 & 30.5 \\
2.5 & 25 & 36.1 \\	
3 & 29 & 37.3 \\
3.5 & 35 & 34.2 \\
4 & 42 & 29.2 \\
4.5 & 46 & 26.1 \\
5 & 48 & 25.4 \\
\bottomrule
\end{tabular}%	
\label{tab:addlabel}%
\end{table}%
ただ、これを書くのはめんどくさいですよね?

なのでみんな大好きExcelを使います。
そこで以下のサイトを参考に、アドインをExcelに追加します。

\url{http://mac-physics.ldblog.jp/archives/51933868.html}

Excelで数字や文字を記入し、罫線等の設定を行った後、導入したアドインを使うと簡単に上のようなものを作成してくれます。
これはすごい便利なので是非活用しよう!。

\section{考察}
もう特に書くことないかな



\forall 
\begin{thebibliography}{99}
\item 誰か著,TeXはいいぞ
\item \url{https://texwiki.texjp.org/},TeX Wiki
\end{thebibliography}

\end{document}